\usepackage{booktabs}
\usepackage{longtable}
\usepackage{framed,color}
\usepackage{amsfonts}
\definecolor{shadecolor}{RGB}{248,248,248}

\ifxetex
  \usepackage{letltxmacro}
  \setlength{\XeTeXLinkMargin}{1pt}
  \LetLtxMacro\SavedIncludeGraphics\includegraphics
  \def\includegraphics#1#{% #1 catches optional stuff (star/opt. arg.)
    \IncludeGraphicsAux{#1}%
  }%
  \newcommand*{\IncludeGraphicsAux}[2]{%
    \XeTeXLinkBox{%
      \SavedIncludeGraphics#1{#2}%
    }%
  }%
\fi

\newenvironment{rmdblock}[1]
  {\begin{shaded*}
  \begin{itemize}
  \renewcommand{\labelitemi}{
    \raisebox{-.7\height}[0pt][0pt]{
      {\setkeys{Gin}{width=3em,keepaspectratio}\includegraphics{images/#1}}
    }
  }
  \item
  }
  {
  \end{itemize}
  \end{shaded*}
  }
\newenvironment{rmdnote}
  {\begin{rmdblock}{note}}
  {\end{rmdblock}}
\newenvironment{rmdcaution}
  {\begin{rmdblock}{caution}}
  {\end{rmdblock}}
\newenvironment{rmdimportant}
  {\begin{rmdblock}{important}}
  {\end{rmdblock}}
\newenvironment{rmdtip}
  {\begin{rmdblock}{tip}}
  {\end{rmdblock}}
\newenvironment{rmdwarning}
  {\begin{rmdblock}{warning}}
  {\end{rmdblock}}



% Use the theorem package
\usepackage{amsthm}

\makeatletter
\def\thm@space@setup{%
  \thm@preskip=8pt plus 2pt minus 4pt
  \thm@postskip=\thm@preskip
}
\makeatother  

\newtheorem{theorem}{Theorem}

\newtheorem{example}{Example}
\let\oldexample\example
\renewcommand{\example}{\oldexample\normalfont}

\newtheorem{remark}{Remark}
\let\oldremark\remark
\renewcommand{\remark}{\oldremark\normalfont}

\newtheorem{lemma}{Lemma}

\newtheorem{corollary}{Corollary}

\newtheorem{proposition}{Proposition}
\newtheorem{definition}{Definition}


\newcommand{\notimplies}{%
  \mathrel{{\ooalign{\hidewidth$\not\phantom{=}$\hidewidth\cr$\implies$}}}}


\DeclareMathOperator*{\mean}{mean}
\DeclareMathOperator*{\var}{var}
\DeclareMathOperator*{\tr}{tr}
\DeclareMathOperator*{\cov}{cov}
\DeclareMathOperator*{\corr}{corr}
\DeclareMathOperator*{\argmax}{argmax}
\DeclareMathOperator*{\argmin}{argmin}
\DeclareMathOperator*{\card}{card}
\DeclareMathOperator*{\diag}{diag}
\DeclareMathOperator*{\rank}{rank}
\DeclareMathOperator*{\length}{length}
\renewcommand{\C}{\text{C}}

% New math symbols
\newcommand{\0}{\boldsymbol{0}}
\newcommand{\I}{\boldsymbol{\mathbf{I}}}
\renewcommand{\S}{\boldsymbol{S}}
\newcommand{\y}{\boldsymbol{y}}
\newcommand{\X}{\boldsymbol{X}}
\newcommand{\bbeta}{\boldsymbol{\beta}}
\newcommand{\bepsilon}{\boldsymbol{\varepsilon}}
\newcommand{\norm}{\mathcal{N}}
\renewcommand{\epsilon}{\varepsilon}
\newcommand{\btheta}{\boldsymbol{\theta}}

\newcommand{\e}[1]{{\mathbb E}\left[ #1 \right]}

\newcommand{\ind}{\boldsymbol{1}}
\newcommand{\normal}{\mathcal{N}}

% Model Selection
\newcommand{\KL}{\text{KL}}
\newcommand{\AIC}{\text{AIC}}
\newcommand{\BIC}{\text{BIC}}



% Different Number sets
\DeclareSymbolFont{lettersA}{U}{txmia}{m}{it}
\DeclareMathSymbol{\real}{\mathord}{lettersA}{"92}
\DeclareMathSymbol{\integers}{\mathord}{lettersA}{154}
\DeclareMathSymbol{\natural}{\mathord}{lettersA}{142}
\DeclareMathSymbol{\rational}{\mathord}{lettersA}{"91}
\DeclareMathSymbol{\irrational}{\mathord}{lettersA}{"90}

\DeclareMathSymbol{\field}{\mathord}{lettersA}{"83}
